%% converted from readme.md
\hypertarget{elevator-systems---optimization}{%
\section{Elevator systems -
optimization}\label{elevator-systems---optimization}}

\hypertarget{abstract}{%
\subsection{Abstract}\label{abstract}}

Imagine we want to construct a building and we want to design an
elevator system for it. How can we do it, so the elevator system is the
most efficient one for this specific building? We run simulations of
different elevator systems and different algorithms, compare them and
pick the best one. This is what this program is about.

\hypertarget{problem}{%
\subsection{Problem}\label{problem}}

Given a building \(B\), we would like to construct an elevator system
for this building \(E_B\), such as \(E_B\) is the most efficient one. We
will measure efficiency by some quality function \(q\).

\hypertarget{definitions}{%
\subsection{Definitions}\label{definitions}}

\hypertarget{simulation}{%
\subsubsection{\texorpdfstring{\textbf{Simulation}}{Simulation}}\label{simulation}}

\begin{itemize}
\tightlist
\item
  It is a discrete simulation, where every step of the simulation
  elevators can either move up, down, stay or board people (these are
  all the events). It uses standard discrete simulation techniques and
  patterns.
\end{itemize}

\hypertarget{attributes}{%
\paragraph{Attributes}\label{attributes}}

\begin{itemize}
\tightlist
\item
  Scheduler
\end{itemize}

\hypertarget{elevator}{%
\subsubsection{\texorpdfstring{\textbf{Elevator}}{Elevator}}\label{elevator}}

\begin{itemize}
\tightlist
\item
  Are controlled by Central Elevator Scheduler
\item
  Elevator in a building. There might be elevators with different
  parameters in the same building, hence each of a different type.
\item
  Elevator doesn't need to have all attributes set. Some elevators can't
  know how many people is on board and knows just the current weight.
  Some others might not even know the current weight.
\end{itemize}

\hypertarget{attributes-1}{%
\paragraph{Attributes}\label{attributes-1}}

\begin{itemize}
\tightlist
\item
  speed
\item
  capacity
\item
  acceleration
\item
  average waiting time of elevator for passengers getting on/off
\item
  current number of people
\item
  current weight
\end{itemize}

\hypertarget{actions}{%
\paragraph{Actions}\label{actions}}

\begin{itemize}
\tightlist
\item
  up()
\item
  down()
\item
  stay()
\item
  board()
\end{itemize}

\hypertarget{building}{%
\subsubsection{\texorpdfstring{\textbf{Building}}{Building}}\label{building}}

\begin{itemize}
\item
  Building where we want our efficient elevator system.
\item
  number of floors
\item
  number of elevators and elevator types
\end{itemize}

\hypertarget{population-distribution}{%
\subsubsection{\texorpdfstring{\textbf{Population
distribution}}{Population distribution}}\label{population-distribution}}

\begin{itemize}
\tightlist
\item
  Assigns each floor how likely a person on this floor would like to use
  an elevator and where probably would the person go, at a given time.
\item
  e.g.~In an office building in the afternoon from office floors it is
  very likely a person would call an elevator and would like to get to
  floor one or underground (parking), because their workday is over, but
  in the morning it will be the other way around (down peak or up peak
  period).
\end{itemize}

\hypertarget{attribues}{%
\paragraph{Attribues}\label{attribues}}

\begin{itemize}
\item
  time

  \begin{itemize}
  \tightlist
  \item
    time of day
  \item
    can be disretized in morning, after lunch, afternoon or in hours,
    minutes, \ldots{}
  \end{itemize}
\item
  each floor has list of probabilities, each corresponding to what floor
  a person might want to get (e.g.~floor 1: 2 - 0.2, 3 - 0.3 4 - 0.2 5 -
  0.2 6 - 0.1)
\item
  each floor has probability of person wanting to use the elevator
  (e.g.~floor 1 - 0.8, floor 2 - 0.05, \ldots{} floor 6 (last) - 0)
\item
  population size

  \begin{itemize}
  \tightlist
  \item
    how many persons can spawn in a day
  \item
    represents total number of people using building's elevators current
    day
  \end{itemize}
\end{itemize}

\hypertarget{situation}{%
\subsubsection{\texorpdfstring{\textbf{Situation}}{Situation}}\label{situation}}

\begin{itemize}
\tightlist
\item
  Represents where (in what floors) are all the elevators and where are
  all the people either waiting for elevator or already in an elevator.
\item
  Central elevator scheduler makes decisions based on the current
  situation.
\item
  Every time an event happens, the current situation changes to next
  situation. Situations are atomic.
\item
  Some attributes might be set or might not. It depends how
  sophisticated you want your elevator system to be. For example, if
  elevator system users have some sort of ID card, than each person can
  call an elevator by the id card and therefore the CES could be certain
  about the number of people in a given floor. In this scenario,
  situation should carry this information. But in a different scenario,
  where users don't have an identification, CES couldn't know how many
  people is actually waiting on each floor. It's only information is how
  many times a button is pressed (and one person can press the button
  how many times he likes), so in this scenario it might not make sense
  to remember people count.
\end{itemize}

\hypertarget{attributes-2}{%
\paragraph{Attributes}\label{attributes-2}}

\begin{itemize}
\tightlist
\item
  list of elevators
\item
  list of floors with people count
\item
  list of floors with indication whether there is a request for elevator
  or not
\end{itemize}

\hypertarget{central-elevator-scheduler}{%
\subsubsection{\texorpdfstring{\textbf{Central Elevator
Scheduler}}{Central Elevator Scheduler}}\label{central-elevator-scheduler}}

\begin{itemize}
\tightlist
\item
  Gives instructions to elevators based on the current situation,
  meaning it assigns every elevator some action (event). Central
  Elevator Scheduler obeys some scheduling algorithm.
\item
  Can use different scheduling algorithms at different times (e.g.~would
  like to use different scheduling algorithm in the morning and in the
  afternoon).
\end{itemize}

\hypertarget{attribues-1}{%
\paragraph{Attribues}\label{attribues-1}}

\begin{itemize}
\tightlist
\item
  population distribution

  \begin{itemize}
  \tightlist
  \item
    it's crucial that CES has this knowledge, because thanks to this, it
    can decide globally and not just locally by the current situation
  \end{itemize}
\item
  situation
\item
  strategy - algorithms to obey at given times
\end{itemize}

\hypertarget{evaluation-function}{%
\subsubsection{\texorpdfstring{\textbf{Evaluation
function}}{Evaluation function}}\label{evaluation-function}}

\begin{itemize}
\tightlist
\item
  evaluates given strategy
\end{itemize}

\hypertarget{how-to-use}{%
\subsection{How to use?}\label{how-to-use}}

You can either run your own simulations based on different parameters,
compare different algorithms and try to optimize it for yourself or you
can use more sophisticated approach and let this program run several
simulations with different algorithms and tweaked parameters to find the
most optimal solution.

\hypertarget{parameters}{%
\subsection{Parameters}\label{parameters}}

These are parameters you are able to set before running the simulation:
1. number of floors in the building 1. number of elevators 1. population
distribution 1. each elevator's parameters 1. CES strategy

\hypertarget{optimization-simulations}{%
\subsection{Optimization simulations}\label{optimization-simulations}}

\hypertarget{what-to-optimize}{%
\subsubsection{What to optimize?}\label{what-to-optimize}}

\begin{itemize}
\item
  It is quite obvious, that the more elevators and the more efficient
  they are the more efficient is our elevator system going to be.
  However, we would like to minimize the number of elevators, because
  each lift schaft is economicaly just a wasted space, that could have
  been used for something more profitable.
\item
  The same goes for elevator parameters (speed, capacity, \ldots{}).
  It's reasonable to keep them bounded below some maximum parameters, to
  make each elevator affordable.
\item
  Hence, it makes sense to try to optimize our elevator system not just
  solely on performance but also on it's cost.
\item
  Neverthless, how good elevator system is will ultimately determined by
  some Evaluation function, that can use completely different evaluating
  principle.
\end{itemize}

\hypertarget{general-simulation}{%
\subsubsection{General simulation}\label{general-simulation}}

\begin{itemize}
\item
  in general simulation, you don't specify number of elevators and their
  qualities.
\item
  general simulation tries to not only optimize strategy, but also
  optimize number of elevators and their parameters to satisfy some
  Evaluation function.
\item
  general simulation just runs multiple concrete strategies with
  different elevator counts and qualities and compares those using some
  quality function.
\item
  You can choose the Evaluation function

  \begin{itemize}
  \tightlist
  \item
    best bet would be quality function based on some price-quality ratio
  \end{itemize}
\end{itemize}

\hypertarget{concrete-simulation}{%
\subsubsection{Concrete simulation}\label{concrete-simulation}}

\begin{itemize}
\item
  you specify concrete number of elevators and their qualities
\item
  concrete simulation just tries to find the best possible strategy for
  given building and population distribution
\end{itemize}

\hypertarget{user-simulation}{%
\subsection{User Simulation}\label{user-simulation}}

\begin{itemize}
\item
  After some optimization simulation has found the best approach how to
  tackle given building and population distribution, it might be
  convenient to try to run a simulation with found optimal elevator
  system and see how it behaves for yourself. This is exactly what user
  simulation is for.
\item
  During user simulation, you can play around and change some
  parameters, to see how adaptive optimal solution is and have feel for
  how it behaves:

  \begin{enumerate}
  \def\labelenumi{\arabic{enumi}.}
  \tightlist
  \item
    spawn persons to exact floors
  \item
    change CES strategy
  \item
    tweak population distribution
  \end{enumerate}
\end{itemize}

\hypertarget{future}{%
\subsection{Future}\label{future}}

\begin{itemize}
\tightlist
\item
  it would be great to not just have very simple scheduling algorithms
  at our disposal but also some more sophisticated techniques, like
  genetic algorithms, machine learning etc \ldots{}
\end{itemize}
